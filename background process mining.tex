\chapter{Background of this study}
\section{Background of this study}
\\

In the life-cycle of a Business process Management Model (BPM) while redesigning the model, actual information about running process was not taken into account. Furthermore, until recently, there were few connections between the data produced while executing the process and the actual process design. Process mining offers the possibility to truly “close” the BPM life-cycle by considering process rich in information.

Process Mining is a young approach connecting machine learning and data mining with the process modelling and analysis to extracting knowledge from event log in aim to do all needed process discovery, monitoring and improve current process. Briefly, it is the link between process data captured from ERP systems (e.g., SAP Business Suite,..etc)
and the process model.
It is interesting to explain process mining in dimensional aspect; the first dimension is types of process mining.

First type is discovery, it means producing model from event log. such technique is used in this thesis when producing Petri-net to explaining process behaviour recorded in the log. Moreover, another algorithm to discover resource model like a social network describing people work in an organisation.

The second type is  conformance. The conformance checking is to check the log conforms to its process model in reality
and vices versa. In other words, if rules are followed or not, a good study of deviations could be achieved.

The third type is enhancement, an extended or improved (enhanced) version on an existing process model could be obtained using running process information. moreover, two types can be mentioned here. Repair i.e.,modifying model to better describing reality. Other is Extension, it means to add a new perspective that can show bottlenecks, throughput times, frequencies,...etc. however, this phase might take a diamond role through this paper and all over case studies.

Process mining gives several perspectives the second dimension, they are the following:
The Control-flow perspective is to characterise all possible paths.

The Organisational perspective to either structure the organisation or showing performing social network.

The Case perspective. As the case is characterised by its unique path along the process or by its number, supplier ..etc.

The Time perspective when discovering bottlenecks, measuring service level as shown in further chapters.

Finally, process mining can be done off-line  on a pre-extracted log as in this thesis, but it might be used in an on-line sitting so-called operational support, that is used for detecting non-conformance when taking place or predicting time for running cases.
\\
\\
Tools: